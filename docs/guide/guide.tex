\documentclass{article}

\usepackage{geometry}
\geometry{legalpaper, margin=1in}

\usepackage{fancyvrb}

\newcommand\iggy{\texttt{iggy}}
\newcommand\optgraph{\texttt{opt\_graph}}

\title{ \iggy  \\
User Guide \\
version 2.0.0
}

\author{Sven Thiele}

\date{}


\begin{document}

\maketitle


\section{What are \iggy\ and \optgraph ?}

\iggy\ and \optgraph\ are tools for consistency based analysis of influence graphs and 
 observed systems behavior (signed changes between two measured states). 
For many (biological) systems are knowledge bases available that describe the interaction
 of its components in terms of causal networks, boolean networks and influence graphs
  where edges indicate either positive or negative effect of one node upon another.

\iggy\  implements methods to check the consistency of large-scale data sets and provides explanations 
for inconsistencies. 
In practice, this is used to identify unreliable data or to indicate missing reactions.
 Further, \iggy\ addresses the problem of repairing networks and corresponding yet often discrepant
  measurements in order to re-establish their mutual consistency and predict unobserved variations 
  even under inconsistency.

\optgraph\ confronts interaction graph models with observed systems behavior from multiple experiments. 
\optgraph\ computes networks fitting the observation data by removing (or adding) a minimal number of edges in the given network.

You can download the precompiled binaries for 64bit linux and macos on the [release page](https://github.com/bioasp/iggy/releases).

You can download the [iggy user guide](https://bioasp.github.io/iggy/guide/guide.pdf).


Sample data is available here: [demo_data.tar.gz](https://bioasp.github.io/iggy/downloads/demo_data.tar.gz)

\section{Compile yourself}

Clone the git repository:

\begin{Verbatim}[frame=single]
 git clone https://github.com/bioasp/iggy.git
 cargo build --release
\end{Verbatim}

The executables can be found under `./target/release/`



\section{Input Model + Data}

\iggy\ and \optgraph\ work with two kinds of data. 
The first is representing an interaction graph model. 
The second is the experimental data, representing experimental condition and observed behavior.

\subsection{Model}

The model is represented as file in complex interaction format \texttt{CIF} as shown below.
Lines in the \texttt{CIF} file specify a interaction between (multiple) source nodes and one target node.

\begin{Verbatim}[frame=single,numbers=left]
 shp2	                         ->  grb2_sos
 !mtor_inhibitor                      ->  mtor
 ?jak2_p                              ->  stat5ab_py
 !ras_gap&grb2_sos                    ->  pi3k
 akt&erk&mtor&pi3k                    ->  mtorc1
 gab1_bras_py	                 ->  ras_gap
 gab1_ps&jak2_p&pi3k&!shp2_ph	 ->  gab1_bras_py
\end{Verbatim}

In our influence graph models we have simple interactions like:
in Line $1$ for \texttt{shp2} \emph{increases} \texttt{grb2\_sos}
and in Line $2$ the \texttt{!} indicates that \texttt{mtor\_inhibitor} tends to \emph{decrease} \texttt{mtor}.
in Line $3$ the \texttt{?} indicates an unknown influence of \texttt{jak2\_p} on \texttt{stat5ab\_py}.
Complex interactions can be composed with the \texttt{\&} operator to model a combined influence of multiple sources on a tartget.
In Line $4$ an decrease in  \texttt{ras\_gap} with an increase in  \texttt{grb2\_sos} tend to increase  \texttt{pi3k}.

 
\subsection{Experimental data}

The experimental data is given in the file format shown below.
Nodes which are perturbed in the experimental condition are denoted as \texttt{input}.
The first line of the example below states that \texttt{depor} has been perturbed in the experiment.
This means \texttt{depor} has been under the control of the experimentalist 
 and its behavior must therefore not be explained.
The behavior of a node can be either \texttt{+}, \texttt{-}, \texttt{0}, \texttt{notPlus}, \texttt{notMinus}.
Line 2 states that an \emph{increase} (\texttt{+}) was obeserved in \texttt{depor},
 as it is declared an \texttt{input} this behavior has been caused by the experimentalist.
Line 3 states that \texttt{stat5ab\_py} has \emph{decreased} (\texttt{-}) and
line 4 states that \texttt{ras} has \emph{not changed} (\texttt{0}).
Line 5 states that an \emph{uncertain decrease} (\texttt{notPlus}) has been observed in \texttt{plcg} and
line 6 states that an \emph{uncertain increase} (\texttt{notMinus}) has been observed in \texttt{mtorc1}.
Line 7 states that \texttt{akt} is initially on the minimum level, this means it cannot further decrease, and
line 8 states that \texttt{gab1} is initially on the maximum level, this means it cannot further increase.

\begin{Verbatim}[frame=single,numbers=left]
depor         = input 
depor         = +
stat5ab_py    = -
ras           = 0
plcg          = notPlus
mtorc1        = notMinus
akt           = MIN
gab1          = MAX
\end{Verbatim}  

\section{Iggy}

Typical usage is:
\begin{Verbatim}[frame=single]
 $ iggy -n network.cif -o observation.obs -l 10 -p
\end{Verbatim}

For more options you can ask for help as follows:
\begin{Verbatim}[frame=single]
 $ iggy -h
 iggy 2.0.0
 Sven Thiele <sthiele78@gmail.com>
 Iggy confronts interaction graph models with observations of (signed) changes between 
 two measured states (including uncertain observations). Iggy discovers inconsistencies
 in networks or data, applies minimal repairs, and predicts the behavior for the 
 unmeasured species. It distinguishes strong predictions (e.g. increase in a node) and
 weak predictions (e.g., the value of a node increases or remains unchanged

 USAGE:
     iggy [FLAGS] [OPTIONS] --network <networkfile> --observations <observationfile>

 FLAGS:
     -a, --autoinputs                Declare nodes with indegree 0 as inputs
         --depmat                    Combine multiple states, a change must be explained
                                     by an elementary path from an input
         --elempath                  Every change must be explained by an elementary 
                                     path from an input
         --founded_constraints_off   Disable foundedness constraints
         --fwd_propagation_off       Disable forward propagation constraints
     -h, --help                      Prints help information
         --mics                      Compute minimal inconsistent cores
         --scenfit                   Compute scenfit of the data, default is mcos
     -p, --show_predictions          Show predictions
     -V, --version                   Prints version information
 OPTIONS:
     -l, --show_labelings <max_labelings>   Show max_labelings labelings, default is OFF,
                                            0=all
     -n, --network <networkfile>            Influence graph in CIF format
     -o, --observations <observationfile>   Observations in bioquali format
\end{Verbatim}

\subsection{Output}

\iggy\ presents the results of its analysis as text output. 
The output of \iggy\ can be redirected into a file using the \texttt{>} operator.
For example to write the results shown below into the file \texttt{myfile.txt} type:

\begin{Verbatim}[frame=single]
$ iggy.py network.sif observation.obs --show_labelings 10 --show_predictions > myfile.txt
\end{Verbatim}

In the following we will dissect the output generated by \iggy.
The first 3 lines of the output state the constraints that have been used to analyze network and data.
For our example it is the default setting with the following constraints.
For a deeper understanding of these constraints see~\cite{sthiele15}.

\begin{Verbatim}[frame=single,numbers=left]
 all observed changes must be explained by an predecessor
 no-change observations must be explained
 all observed changes must be explained by an input
\end{Verbatim}

Next follow some statistics on the input data. 
Line 4-5 tells us that the influence graph model given as \texttt{network.sif} 
consists of \texttt{96} nodes, 
with \texttt{116} edges with activating influence 
and \texttt{16} edges with inhibiting influence
and \texttt{0} edges with \texttt{Dual} or ambiguous influence.
%
Line 9 tells that the experimental data given as \texttt{observation.obs} in itself is \texttt{consistent},
which means it does not contain contradictory observations.
Line 11 tells that the experimental conditions consists of \texttt{14} perturbations marked as \texttt{input} nodes, 
that \texttt{12} nodes were observed as increased \texttt{+}, 
\texttt{10} nodes \emph{decreased} (\texttt{-}),
\texttt{20} nodes did \emph{not change} (\texttt{0}),
\texttt{5} nodes were observed with an \emph{uncertain decrease} (\texttt{notPlus}),
\texttt{4} nodes were observed with an \emph{uncertain increase} (\texttt{notMinus}),
\texttt{74} nodes were \texttt{unobserved} and the experimental data contained \texttt{0} observations of things that are not in the given influence graph model.

\begin{Verbatim}[frame=single,numbers=left,firstnumber=4] 
Reading network network.sif ... done.
   Nodes: 96  Activations: 116  Inhibitions: 16  Dual: 0

Reading observations observation.obs ... done.

Checking observations observation.obs ... done.
   Observations are consistent.
   inputs: 14  observed +: 12  observed -: 10  observed 0: 20  observed notPlus: 5  
   observed notMinus: 4 unobserved: 74  not in model: 0
\end{Verbatim}

Then follow the results of the consistency analysis.
Line 14 tells us that network and data are inconsistent 
and that the size of a \emph{minimal correction set} (\texttt{mcos}) is \texttt{1}.
This means that at least \texttt{1} influence needs to be added to restore consistency.
For a deeper understanding of mcos see~\cite{samaga13a}.
Further the output contains at most \texttt{10} consistent labeling including correction set. 
This is because we choose to set the flag \texttt{--show\_labelings 10}.
In our example we have only \texttt{2} possible labelings. 
Each labeling represents a consistent behavior of the model (given mcos the corrections).
\texttt{Labeling 1},
tells it is possible that 
\texttt{STAT3\_n} and \texttt{PAK1} \emph{increase} (\texttt{+}),
\texttt{IGF1\_act} does \emph{not change} (\texttt{0}) and that
\texttt{KS6A5/KS6A4} and \texttt{TNR1A/TNR1B} \emph{decrease} (\texttt{-}).
Line 26 tells us that this is a consistent behavior if \texttt{MTOR} would receive a positive influence, 
which is currently not included in the model.
\texttt{Labeling 2}, represents an alternative behavior,
 here  \texttt{PAK1} and \texttt{KS6A5/KS6A4} do \emph{not change} (\texttt{0}).
Please note that in this example both labelings are consistent under the same correction set.
In another example more than one minimal correction set could exists.

\begin{Verbatim}[frame=single,numbers=left,firstnumber=13]  
Computing mcos of network and data ... done.
   The network and data are inconsistent: mcos = 1.
  
Compute mcos labelings ... done.
Labeling 1:
   gen("STAT3_n") = +
   gen("PAK1") = +
   gen("IGF1_act") = 0
   gen("KS6A5/KS6A4") = -
   gen("TNR1A/TNR1B") = -

   labeled +: 2  labeled -: 2  labeled 0: 1 

   new_influence("observation.obs",gen("MTOR"),1)
   
Labeling 2:
   gen("STAT3_n") = +
   gen("PAK1") = 0
   gen("IGF1_act") = 0
   gen("KS6A5/KS6A4") = 0
   gen("TNR1A/TNR1B") = -

   labeled +: 1  labeled -: 1  labeled 0: 3 
   
   new_influence("observation.obs",gen("MTOR"),1)
\end{Verbatim}

Finally the prediction results are listed.
A prediction is a statement that hold under all labeling under all minimal repairs. 
For a formal definition of predictions see~\cite{sthiele15}.
Here the predictions say that 
\texttt{STAT3\_n} \emph{always increases} (\texttt{+}),
\texttt{PAK1} \emph{never decreases} (\texttt{NOT -}),
\texttt{IGF1\_act} always stays \emph{unchanged} (\texttt{0}),
\texttt{KS6A5/KS6A4} \emph{never increases} (\texttt{NOT +}), and that
\texttt{TNR1A/TNR1B} \emph{always decreases} (\texttt{-}).

\begin{Verbatim}[frame=single,numbers=left,firstnumber=38]  
Compute predictions under mcos ... done.
   gen("STAT3_n") = +
   gen("PAK1") = NOT -
   gen("IGF1_act") = 0
   gen("KS6A5/KS6A4") = NOT +
   gen("TNR1A/TNR1B") = -
 
   predicted +: 1  predicted -: 1  predicted 0: 1  predicted NOT +: 1  
   predicted NOT -: 1  predicted CHANGE: 0
\end{Verbatim}  

\bibliographystyle{plain}
\bibliography{local} 
 
\end{document}

\section{Opt\_graph}

Typical usage is:

\begin{Verbatim}[frame=single]
 $ opt_graph -n network.cif -o observations_dir/ --show_repairs 10
\end{Verbatim}

For more options you can ask for help as follows:

\begin{Verbatim}[frame=single]
 $ opt_graph -h
 opt_graph 2.0.0
 Sven Thiele <sthiele78@gmail.com>
 Opt-graph confronts interaction graph models with observations of (signed) changes between two measured 
 states. Opt-graph computes networks fitting the observation data by removing (or adding) a minimal number 
 of edges in the given network

 USAGE:
     opt_graph [FLAGS] [OPTIONS] --network <networkfile> --observations <observationdir>
 FLAGS:
     -a, --autoinputs                Declare nodes with indegree 0 as inputs
         --depmat                    Combine multiple states, a change must be explained by an                                elementary path from an input
         --elempath                  Every change must be explained by an elementary path from an                             input
         --founded_constraints_off   Disable foundedness constraints
         --fwd_propagation_off       Disable forward propagation constraints
     -h, --help                      Prints help information
     -V, --version                   Prints version information
 OPTIONS:
     -r, --show_repairs <max_repairs>      Show max_repairs repairs, default is OFF, 0=all
     -n, --network <networkfile>           Influence graph in CIF format
     -o, --observations <observationdir>   Directory of observations in bioquali format
     -m, --repair_mode <repair_mode>       Repair mode: remove = remove edges (default),
                                                        optgraph = add + remove edges,
                                                        flip = flip direction of edges
\end{Verbatim}







